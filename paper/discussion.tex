\section{Discussion}

\label{discussion}

We have developed a new approach to estimate SFHs of galaxies from their optical absorption spectra, using simulation-based inference to obtain posteriors in a fully probabilistic treatment. From interpretable latent representations of the spectra, we predict the stellar mass  growth and metallicity, reaching high accuracy for the synthetic sample, as well as  properly calibrated uncertainties, evaluated through a SBC test. The predictions require less than half a second for each galaxy, sampling the posteriors with $1{,}000$ evaluations. This not only makes it possible to address a large dataset, but also to increase the number of evaluations for the posteriors, raising the precision of both the measurements and their uncertainties. \\


\subsection{Model considerations}

Although our new model is substantially faster than other Bayesian inference methods, such as MCMC, and provides well-calibrated uncertainties, unlike classical inversion methods applied to spectral fitting, it is subject to different systematics and modelling assumptions. One of the obvious limitations is the use of a forward model to train the network, because as our understanding of galaxy evolution is incomplete, the considerations taken when generating the synthetic data will never be fully constrained by a typical galaxy spectrum. Therefore, the prior, set by the distribution of parameters in the training set, will always have at least a moderate role in determining the answer. In particular, the simplifications made in terms of chemical evolution, by combining MILES SSP spectra with a fixed metallicity (instead of taking into account the metallicity histories) and working with base $\alpha$-enhancement models, may affect the reliability of our model predictions, given the intrinsic degeneration between the age and the metal content in the spectra. Moreover, there is currently no consensus on the stellar evolution \citep{Conroy_2009} or on the IMF of galaxies \citep{Mart_n_Navarro_2014}, and our model training has been restricted to a specific set of isochrones assuming a particular IMF parametrization. This limits the range of the training data, and dooms our model to fail in observations of galaxies that differ from these considerations.\\

It is known that the SFHs have a strong impact on the predicted posteriors, as shown by \cite{Leja_2019}, and models that mimic the breadth  distribution of SFR($t$) in the real Universe are required. In this work, in the pipeline provided by the GP-SFH module, we select a dispersive prior: a Dirichlet distribution with $\alpha=1.0$ for the fractional sSFR in each of the three equally-spaced time bins, positioning ourselves in favour of short-term variations in the SFHs (with smaller characteristic times). However, this selection must be explored with caution in future works.\\

On the other hand, a tuning of the backward model hyperparameters could be performed, among which we highlight the number of components of the latent representations in which we encode the spectrum, as well as the number of blocks used in the Masked Autoregressive Flow.


\subsection{Observations}

We obtain measurements of the SFH and the metallicity for stacks of SDSS spectra of ETGs. The model recovers the well-known relationship between  age and velocity dispersion, showing that the most massive galaxies ($\sigma \sim 300$ km/s) build up their stellar masses more abruptly, up to 90\% of their total stellar mass $1$ Gyr after the Big Bang, while the growth of stellar mass is softened in less massive galaxies. How these massive galaxies, up to $M_{*} \sim 10^{12} \, \rm M_{\odot}$ for the highest velocity dispersion stack, can form most of their masses so early is a question that is still open, given that a priori it requires a very high star formation efficiency. These galaxies are rare in the known Universe, with rapid bursts of star formation and followed by rapid quenching. The stars we see may have formed in different progenitor galaxies, which were located in the highest-density peaks of the Cosmic Web, and later assembled in major mergers \citep{conselice2007assembly}. To attend to these processes, hierarchical assembly models are necessary, commonly studied through large-scale cosmological simulations \citep[e.g.][]{Angeloudi_2023}. \\


As future work, we propose to measure SFHs and metallicities for late-type galaxies, for which we expect modelling difficulties due to the larger number of emission lines and potential `over-shining' effects when recovering the metallicity and ages of their first stars to form. So far we have not incorporated emission lines, noise models or dust attenuation, but including these aspects in the future is essential to manage a wide range of observations.  We can model the emission lines with specialised libraries \citep[e.g.][]{cloudy,fado}, or with data-driven approaches, training a neural network to  directly learn the relation between fundamental physical properties, such as stellar mass, SFR and metallicity, and observed emission features in real spectra. This last method can  also be extended to apply realistic noise to the spectra in the simulation, possibly introducing both the noisy spectra and the uncertainty distributions for the fluxes into the encoder, resulting in noise-aware latent representations.  In this way, we can condition the Bayesian inference with the noise.\\

Our approach can contribute, given its speed and error handling, to study the biases that produce features of the forward model that have traditionally been assumed in similar analyses. Following the first steps taken in this project for the stacks of SDSS spectra of ETGs, we plan to expand our model to  perform a deep sampling of current or upcoming spectroscopic surveys such as DESI \citep{desi}, WEAVE \citep{weave}, 4MOST \citep{4most}, or MOONS \citep{MOONS}, which will observe billions of galaxies and produce more than $10^8$ TB of data \citep{2023Smith}. 
