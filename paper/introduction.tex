\section{Introduction}

 Understanding the physical processes that regulate star formation over cosmic time is one of the main challenges of galaxy studies, as their evolution depends on a balance between processes that trigger star formation and others that prevent it by expelling or heating gas \citep[e.g.][]{Franx90,Silk_2012, lilly13, nacho}. Reconstructing SFHs is thus a fundamental step towards understanding galaxy evolution. However, inferring  from observed spectra for a statistically significant sample of galaxies is a complex inverse problem subject to a large number of degeneracies which are not well understood \citep[e.g.][]{worthey92, Ocvirk, Conroy_2009}. \\

  Stellar population synthesis (SPS) models are normally used, the building blocks of which are simple stellar populations (SSPs). SSPs describe the evolution in time of the spectrum of a single stellar burst of fixed metallicity and chemical abundance, requiring three basic inputs: stellar evolution theory in the form of isochrones, e.g., Padova \citep{Girardi_2000}, BaSTI \citep{Pietrinferni_2006}, or MIST \citep{Choi_2016};  empirical or  theoretical stellar spectral libraries, e.g., CaT \citep{Cenarro_2001}, MILES \citep{Vazdekis_2010}, or XSL \citep{Gonneau_2020}; and an IMF \citep{salpeter95,Kroupa_2001,Chabrier_2003}, each of which may in principle depend on metallicity and/or elemental abundance patterns.  Composite stellar populations (CSPs) differ from simple ones because they contain stars with a range of ages given by their SFHs and with a range of metallicities as given by their time-dependent metallicity distribution function $P$($\rm{[M/H]}$\footnote{i.e. metals over hydrogen, defined as $\rm{[M/H]}=\log \left(\frac{\rm M}{\rm H}\right)-\log \left(\frac{\rm M}{\rm H}\right)_{\odot}$.}, $t$).\\



To connect a model with observations, a statistical inference framework is required. Full spectral fitting algorithms are a popular tool to infer stellar population properties from integrated spectra, generally based on a backward comparison between data and models  \citep[e.g.][]{starlight,Ocvirk,vespa,firefly, Cappellari2022}. The success and reliability of this method depends on the quality of the template spectra, and of the robustness of the fitting algorithm. In parallel, stellar population inversion algorithms have evolved into Bayesian statistics \citep[e.g.][]{Mart_n_Navarro_2019,Johnson_2021,2024maciata, 2024wang}, using primarily the Markov Chain Monte Carlo (MCMC) method for sampling the posterior distributions, which are assumed to be Gaussian, allowing an efficient exploration of the degeneracies associated with the large parameter space.\\ 

Existing Bayesian spectral modelling approaches, however, require a substantial computational investment, ranging from $10$ - $100$ CPU hours per galaxy \citep{Carnall_2019,tacchela21}. Although this computational demand is already high for current datasets, like SDSS \citep{SDSS} or COSMOS \citep{cosmos}, containing hundreds of thousands of galaxy spectral energy distributions (SEDs), it becomes a prohibitive bottleneck for upcoming surveys. Over the next decade, DESI \citep{desi}, WEAVE \citep{weave}, 4MOST \citep{4most}, and MOONS \citep{MOONS} will capture spectra from billions of galaxies, which would imply tens or hundreds of billions of CPU hours.\\

Machine learning-based models are becoming more popular in astronomy \citep[e.g.][]{huertascompany2023brief, joanna24,moser24,hunt2024predicting}. In contrast with traditional spectral fitting, where SFHs are built from some ensemble of SSPs to recreate the spectra, machine learning directly learns the relationship between the observed features and the entire SFHs. Its systematic uncertainties are thus independent of those from spectral fitting, complementing and strengthening the measurements \citep{lovell19}. On the other hand, neural density estimators have recently been introduced to address the computational challenge of Bayesian methods \citep[e.g.][]{kwon24}, demonstrating a significant acceleration in SED model evaluations of up to five orders of magnitude \citep{Hahn_2022}, so the posterior inference for each galaxy now takes seconds.\\


In this work, we explore a novel approach based on probabilistic machine learning and likelihood-free inference to estimate the SFHs of galaxies from their optical absorption spectra.  The main advantage over classical Bayesian inference methods is that our approach does not assume a functional form for likelihood, typically considered Gaussian, and that once the model is trained, it can be evaluated on different observations with minimal computational cost. The outline of the paper is as follows: in Sect.~\ref{methods} we describe the Bayesian inference framework, using SPS (\ref{forward}), an encoder of the spectra (\ref{encoder}),  and a neural density estimator (\ref{normflows}). In Sect.~\ref{results} we test the model in both mocked observations and stacks of early-type galaxies from SDSS. We further discuss the details of the model and the measurements of the stacks' physical properties in Sect.~\ref{discussion}.
